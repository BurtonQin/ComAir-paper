\section{Understanding Real-World Complexity Bugs}
\label{sec:study}


In this section, we conduct an empirical study on real-world complexity bugs. 
Specifically, we try to answer the following two questions:

\begin{enumerate}

\item What are common root causes for complexity bugs? 
Here, we consider code constructs conducting computation in unexpected complexity as root causes. 
Studying root causes can help us understand whether or not 
existing algorithmic profiling algorithms work on real-world bugs 
and can also guide the design of new approximate algorithms. 

\item How real-world complexity bugs are reported, diagnosed, and fixed?
This can help us understand the state of practice of combating process for complexity bugs. 

\end{enumerate}


\subsection{Methodology}
\label{sec:meth}

We conduct our empirical study based on a public benchmark set for 
real-world performance bugs~\cite{PerfBug,SongOOPSLA2014}. 
In the previous work~\cite{PerfBug}, the authors collected 110 real-world 
performance bugs from 5 representative 
software suites, including Apache, Chrome, GCC, Mozilla, and MySQL. 
These software suites cover various types of functionalities and 
are implemented by different programming languages, 
like C/C++, Java, C\#, and JavaScript. 
These software suites are also large and mature, 
with millions of lines of codes and long development histories. 
In their following work~\cite{SongOOPSLA2014}, 
the authors further identified 65 user-perceived performance bugs, 
which are all observed and reported by users. 


Our study focuses on user-perceived performance bugs, 
since these bugs have large performance impact.
We carefully read bug reports and buggy codes for these bugs, 
and clearly identify \ComBugs bugs caused by {\textit{unexpected algorithmic complexity}},
and will be referred to as {\textit{complexity bugs}} in the reminder of this paper. 
We believe the collected complexity bugs can serve as a representative sample 
of complexity bugs in the real world. 
The detailed distribution for studied complexity bugs is shown in Table~\ref{xx}. 


