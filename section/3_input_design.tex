\section{Technique Design}
\label{sec:design}

\subsection{Input size metric}


\subsection{Cost metric}

\subsubsection{Executed basic blocks (BBs)}
The number of executed BBs for a dynamic instance of a code construct
can be used to measure the execution cost for the dynamic instance. 
A naive method to collect this metric is to add a global counter 
to the monitored program and increase the counter value by 1 inside each BB.  
When entering and leaving a monitored code construct, 
the value of the counter will be dumped to log.

To efficiently count executed BBs, 
we apply an algorithm, which was originally designed to 
efficiently count edge events through selectively instrumenting a counter 
on CFG~\cite{event-counting}.
The algorithm has already been proved to be able to 
conduct path profiling efficiently~\cite{peter-ase,path-profiling}. 



To apply the algorithm,
we instrument a local counter \texttt{local\_cost} for each function
and initialize its value to be 0 at the entry BB. 
We will add the value of \texttt{local\_cost} to a global counter \texttt{cost} 
at the exit BB.
After that, we only need to consider where 
and how to update \texttt{local\_cost} 
within a single function.
We design and implement an intra-procedural control flow analysis
to achieve this.
Given a single function,
we add a fake edge from the exit BB to the entry BB 
to make its CFG strongly connected. 
Since the original algorithm is design to count edge events,
we split each BB into two 
and label the event number to be 1 for each edge connecting a pair of split BBs 
We label the event number to be 0 for all other edges.
We compute a spanning tree~\cite{spanning} for the new CFG.
Edges not in the spanning tree are called chords.
We apply the depth-first search algorithm proposed in~\cite{event-counting} 
to calculate on which chords we should change 
\texttt{local\_cost} 
and how much we should change.

If a monitored code construct is not a function,
We need to figure out the accurate value for the global counter \texttt{cost} 
before dumping its value 
at the first and last BB of the monitored code construct.
The algorithm we apply is also discussed in~\cite{event-counting}.




\subsection{Curve}