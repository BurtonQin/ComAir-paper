
%\begin{table*}[tb!]
%\begin{adjustwidth}{-.5in}{-.5in}
%\small
%\centering
%{
%\begin{tabular}{|lcccccc|}
%\hline
%                                                               &   Apache  &   Chrome   &  GCC   &    Mozilla   &   MySQL  &  Total\\
%\hline
%Total \# of complexity bugs                                          &   4       &    3       &   8    &    10        &   5      &   30 \\
%\hline
%\multicolumn{7}{|c|}{\bf Taxonomy of complexity problems}\\
%\multicolumn{1}{|l}{{\bf $O(N)$}: linear complexity}                 &   1       &    0       &   0    &    4         &   3      &   8\\
%\multicolumn{1}{|l}{{\bf $O(N^k)$}: polynomial complexity (k>1)}     &   3       &    3       &   5    &    6         &   1      &  18\\
%\multicolumn{1}{|l}{{\bf $O(e^N)$}: exponential complexity}          &   0       &    0       &   3    &    0         &   1      &   4\\
%\hline
%\multicolumn{7}{|c|}{\bf How complexity problems are reported?}\\
%\multicolumn{1}{|l}{How to change input size {\bf is} specified}     &  4&2&5&9&5&25\\
%\multicolumn{1}{|l}{How to change input size {\bf not} specified}    &  0&1&3&1&0&5\\
%\hline
%\end{tabular}
%}
%\end{adjustwidth}
%\caption{Categorization for Section~\ref{sec:study}.
%(This table shows how complexity problems distribute among different complexity categories 
% and whether or not how to change input size is specified during reporing.)}
%\label{tab:study}
%\end{table*}


\begin{table}[tb!]
\begin{adjustwidth}{-.5in}{-.5in}
\small
\centering
{
\begin{tabular}{|lcccccc|}
\hline
                                                                                  &   Apache  &   Chrome   &  GCC   &    Mozilla   &   MySQL  &  Total\\
\hline
Total \# of complexity bugs                                                       &   4       &    3       &   8    &    10        &   5      &   30 \\
\hline
\multicolumn{7}{|c|}{\bf Taxonomy of complexity problems}\\
\multicolumn{1}{|l}{{\bf $O(N)$}: linear complexity}                              &   1       &    0       &   0    &    4         &   2      &   7\\
\multicolumn{1}{|l}{{\bf $O(N^k)$}: polynomial complexity ($k>1$)}                &   3       &    3       &   5    &    6         &   2      &  19\\
\multicolumn{1}{|l}{{\bf $O(e^N)$}: exponential complexity}                       &   0       &    0       &   3    &    0         &   1      &   4\\
\hline
\multicolumn{7}{|c|}{\bf How complexity problems are fixed?}\\
\multicolumn{1}{|l}{Optimize {\bf directly}: Buggy code is revised}              &  3        &    3       &   4    &    9         &   5      &  24 \\
\multicolumn{1}{|l}{Optimize {\bf indirectly}: Workloads are skipped}              &  1        &    0       &   4    &    1         &   0      &   6\\
\hline
\multicolumn{7}{|c|}{\bf How complexity problems are reported?}\\
\multicolumn{1}{|l}{Input size {\bf is} specified}                                &  4        &    2       &   4    &    9    &5   &24\\
\multicolumn{1}{|l}{Input size is {\bf not} specified}                            &  0        &    1       &   4    &    1    &0   &6\\
\hline
\end{tabular}
}
\end{adjustwidth}
\caption{Categorization for Section~\ref{sec:study}.
\footnotesize{(This table shows how complexity problems distribute among different complexity categories, how developers fix studied complexity problems,  
 and whether or not how to change input size is specified during reporting.)}}
\label{tab:study}
\vspace{-0.4in}
\end{table}