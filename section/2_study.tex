\section{Understanding Real-World Complexity Bugs}
\label{sec:study}


In this section, we conduct an empirical study on real-world complexity bugs. 
Specifically, we try to answer the following two questions:

\begin{enumerate}

\item What are common root causes for complexity bugs? 
Here, we consider code constructs conducting computation in unexpected complexity as root causes. 
Studying root causes can help us understand whether or not 
existing algorithmic profiling algorithms work on real-world bugs 
and can also guide the design of new approximate algorithms. 

\item How real-world complexity bugs are reported, diagnosed, and fixed?
This can help us understand the state of practice of combating process for complexity bugs. 

\end{enumerate}


\subsection{Methodology}
\label{sec:meth}

We conduct our empirical study based on a public benchmark set for 
real-world performance bugs~\cite{PerfBug,SongOOPSLA2014}. 
In the previous work~\cite{PerfBug}, the authors collected 110 real-world 
performance bugs from 5 representative 
software suites, including Apache, Chrome, GCC, Mozilla, and MySQL. 
These software suites cover various types of functionalities and 
are implemented by different programming languages, 
like C/C++, Java, C\#, and JavaScript. 
These software suites are also large and mature, 
with millions of lines of codes and long development histories. 
In their following work~\cite{SongOOPSLA2014}, 
the authors further identified 65 user-perceived performance bugs, 
which are all observed and reported by users. 


Our study focuses on user-perceived performance bugs, 
since these bugs have large performance impact.
After carefully reading bug reports and buggy code fragments for these bugs,
we clearly identify \ComBugs bugs caused by {\textit{unexpected algorithmic complexity}}.
These bugs will be referred to as {\textit{complexity bugs}} in the reminder of this paper.
The detailed distribution for studied complexity bugs is shown in Table~\ref{tab:study}. 


{\bf{\textit{Caveats}}}
The same as all previous empirical studies, 
our findings and conclusions need to be considered with our methodology.
All studied complexity bugs come from representative software. 
However, there are other types of software not covered in our study, 
such as distributed systems and HPC software. 
All studied complexity bugs are reported by users through bug tracking systems.  
There could be complexity bugs noticed and fixed by developers in other ways. 
However, there are no conceivable methods to study them.
We believe the studied complexity bugs can serve as a good representative sample
of complexity bugs in the real world. 

\section{Root causes of complexity bugs}


%\begin{table*}
%\begin{adjustwidth}{-.5in}{-.5in}
%\small
%\centering
%{
%\begin{tabular}{|lcccccc|}
%\hline
%                                                               &   Apache  &   Chrome   &  GCC   &    Mozilla   &   MySQL  &  Total\\
%\hline
%Total \# of complexity bugs                                          &   4       &    3       &   8    &    10        &   5      &   30 \\
%\hline
%\multicolumn{7}{|c|}{\bf Taxonomy of complexity problems}\\
%\multicolumn{1}{|l}{{\bf $O(N)$}: linear complexity}                 &   1       &    0       &   0    &    4         &   3      &   8\\
%\multicolumn{1}{|l}{{\bf $O(N^k)$}: polynomial complexity (k>1)}     &   3       &    3       &   5    &    6         &   1      &  18\\
%\multicolumn{1}{|l}{{\bf $O(e^N)$}: exponential complexity}          &   0       &    0       &   3    &    0         &   1      &   4\\
%\hline
%\multicolumn{7}{|c|}{\bf How complexity problems are reported?}\\
%\multicolumn{1}{|l}{How to change input size {\bf is} specified}     &  4&2&5&9&5&25\\
%\multicolumn{1}{|l}{How to change input size {\bf not} specified}    &  0&1&3&1&0&5\\
%\hline
%\end{tabular}
%}
%\end{adjustwidth}
%\caption{Categorization for Section~\ref{sec:study}.
%(This table shows how complexity problems distribute among different complexity categories
% and whether or not how to change input size is specified during reporing.)}
%\label{tab:study}
%\end{table*}


\begin{table*}[tb!]
\begin{adjustwidth}{-.5in}{-.5in}
\small
\centering
{
\begin{tabular}{|lcccccc|}
\hline
                                                                                  &   Apache  &   Chrome   &  GCC   &    Mozilla   &   MySQL  &  Total\\
\hline
Total \# of complexity bugs                                                       &   4       &    3       &   8    &    10        &   5      &   30 \\
\hline
\multicolumn{7}{|c|}{\bf Taxonomy of complexity problems}\\
\multicolumn{1}{|l}{{\bf $O(N)$}: linear complexity}                              &   1       &    0       &   0    &    4         &   2      &   7\\
\multicolumn{1}{|l}{{\bf $O(N^k)$}: polynomial complexity ($k>1$)}                &   3       &    3       &   5    &    6         &   2      &  19\\
\multicolumn{1}{|l}{{\bf $O(e^N)$}: exponential complexity}                       &   0       &    0       &   3    &    0         &   1      &   4\\
\hline
\multicolumn{7}{|c|}{\bf How complexity problems are fixed?}\\
\multicolumn{1}{|l}{Optimize {\bf directly}: Buggy code is revised}              &  3        &    3       &   4    &    9         &   5      &  24 \\
\multicolumn{1}{|l}{Optimize {\bf indirectly}: Workloads are skipped}              &  1        &    0       &   4    &    1         &   0      &   6\\
\hline
\multicolumn{7}{|c|}{\bf How complexity problems are reported?}\\
\multicolumn{1}{|l}{Input size {\bf is} specified}                                &  4        &    2       &   4    &    9    &5   &24\\
\multicolumn{1}{|l}{Input size is {\bf not} specified}                            &  0        &    1       &   4    &    1    &0   &6\\
\hline
\end{tabular}
}
\end{adjustwidth}
\caption{Categorization for Section~\ref{sec:study}.
\footnotesize{(This table shows how complexity problems distribute among different complexity categories, how developers fix studied complexity problems,
 and whether or not how to change input size is specified during reporting.)}}
\label{tab:study}
% \vspace{-0.4in}
\end{table*}

