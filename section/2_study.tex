\section{Understanding Real-World Complexity Problems}
\label{sec:study}

In this section, we conduct an empirical study on real-world complexity problems. 
Specifically, we try to answer the following two questions:

\begin{enumerate}

\item What are common root causes for complexity bugs? 
Here, we refer root cause as a code construct conducting 
inefficient computation leading to a complexity problem.
Studying root causes can help us 
make suitable design decisions for algorithmic profiling. 

\item How real-world complexity problems are reported and diagnosed?
This knowledge can help us understand the 
state of practice for combating complexity problems.

\end{enumerate}

\subsection{Methodology}
\label{sec:meth}

We conduct our empirical study based on a public benchmark set for 
real-world performance bugs~\cite{PerfBug,SongOOPSLA2014}. 
In the previous work~\cite{PerfBug}, 
the authors collected 110 real-world performance bugs from 5 representative 
software suites: Apache, Chrome, GCC, Mozilla, and MySQL. 
These software suites cover various types of functionalities and are implemented 
by a variety of programming languages, including C/C++, Java, C\#, and JavaScript. 
All five studied software suites are large and mature, 
with millions of lines of codes and long development histories. 
In their following work~\cite{SongOOPSLA2014}, 
the authors further identified 65 user-perceived performance bugs, 
which are all observed and reported by users. 

Our study focuses on user-perceived performance bugs, 
since these bugs have a large performance impact.
After carefully reading bug reports and buggy code fragments 
associated with these bugs,
we clearly identify \ComBugs performance problems 
caused by {\textit{unexpected algorithmic complexity}}.
These performance problems will be referred to as 
{\textit{complexity problems}} in the reminder of this paper.

{\bf{\textit{Caveats.}}}
In keeping with all previous empirical studies, 
our findings and conclusions need to be considered together with our methodology.
All studied complexity bugs come from software representative of a variety of uses and development processes. 
However, there are other types of software not covered in our study, 
such as distributed systems and high performance computation software. 
All studied complexity problems are reported by users through bug tracking systems.  
We do believe that there are complexity problems noticed 
and fixed by developers through other processes. 
%However, there are no conceivable methods to study them.
However, the field has not found methods to study them.
We believe that the studied complexity problems can serve as a representative sample
of complexity problems in the real world. 


\subsection{Root causes of complexity problems}
\label{sec:tax}


%\begin{table*}
%\begin{adjustwidth}{-.5in}{-.5in}
%\small
%\centering
%{
%\begin{tabular}{|lcccccc|}
%\hline
%                                                               &   Apache  &   Chrome   &  GCC   &    Mozilla   &   MySQL  &  Total\\
%\hline
%Total \# of complexity bugs                                          &   4       &    3       &   8    &    10        &   5      &   30 \\
%\hline
%\multicolumn{7}{|c|}{\bf Taxonomy of complexity problems}\\
%\multicolumn{1}{|l}{{\bf $O(N)$}: linear complexity}                 &   1       &    0       &   0    &    4         &   3      &   8\\
%\multicolumn{1}{|l}{{\bf $O(N^k)$}: polynomial complexity (k>1)}     &   3       &    3       &   5    &    6         &   1      &  18\\
%\multicolumn{1}{|l}{{\bf $O(e^N)$}: exponential complexity}          &   0       &    0       &   3    &    0         &   1      &   4\\
%\hline
%\multicolumn{7}{|c|}{\bf How complexity problems are reported?}\\
%\multicolumn{1}{|l}{How to change input size {\bf is} specified}     &  4&2&5&9&5&25\\
%\multicolumn{1}{|l}{How to change input size {\bf not} specified}    &  0&1&3&1&0&5\\
%\hline
%\end{tabular}
%}
%\end{adjustwidth}
%\caption{Categorization for Section~\ref{sec:study}.
%(This table shows how complexity problems distribute among different complexity categories
% and whether or not how to change input size is specified during reporing.)}
%\label{tab:study}
%\end{table*}


\begin{table*}[tb!]
\begin{adjustwidth}{-.5in}{-.5in}
\small
\centering
{
\begin{tabular}{|lcccccc|}
\hline
                                                                                  &   Apache  &   Chrome   &  GCC   &    Mozilla   &   MySQL  &  Total\\
\hline
Total \# of complexity bugs                                                       &   4       &    3       &   8    &    10        &   5      &   30 \\
\hline
\multicolumn{7}{|c|}{\bf Taxonomy of complexity problems}\\
\multicolumn{1}{|l}{{\bf $O(N)$}: linear complexity}                              &   1       &    0       &   0    &    4         &   2      &   7\\
\multicolumn{1}{|l}{{\bf $O(N^k)$}: polynomial complexity ($k>1$)}                &   3       &    3       &   5    &    6         &   2      &  19\\
\multicolumn{1}{|l}{{\bf $O(e^N)$}: exponential complexity}                       &   0       &    0       &   3    &    0         &   1      &   4\\
\hline
\multicolumn{7}{|c|}{\bf How complexity problems are fixed?}\\
\multicolumn{1}{|l}{Optimize {\bf directly}: Buggy code is revised}              &  3        &    3       &   4    &    9         &   5      &  24 \\
\multicolumn{1}{|l}{Optimize {\bf indirectly}: Workloads are skipped}              &  1        &    0       &   4    &    1         &   0      &   6\\
\hline
\multicolumn{7}{|c|}{\bf How complexity problems are reported?}\\
\multicolumn{1}{|l}{Input size {\bf is} specified}                                &  4        &    2       &   4    &    9    &5   &24\\
\multicolumn{1}{|l}{Input size is {\bf not} specified}                            &  0        &    1       &   4    &    1    &0   &6\\
\hline
\end{tabular}
}
\end{adjustwidth}
\caption{Categorization for Section~\ref{sec:study}.
\footnotesize{(This table shows how complexity problems distribute among different complexity categories, how developers fix studied complexity problems,
 and whether or not how to change input size is specified during reporting.)}}
\label{tab:study}
% \vspace{-0.4in}
\end{table*}


In this section, we study root causes of complexity problems in different categories. 
Our study focuses on 
1) what code constructs causing the complexity problems, 
2) why the complexity problems can generate user-perceived performance impact, 
and 3) how developers fix these complexity problems. 


{\underline{\textit{$O(N)$: linear complexity.}}} 
As shown in Table~\ref{tab:study}, 
8 out of \ComBugs studied complexity problems are in linear complexity. 
All of these problems are caused by a buggy loop, 
whose average loop iteration number scales in terms of input size $N$.

For 5 of them, their corresponding buggy loops contain serialized I/O operations.
Although average loop iteration numbers or input sizes are not large,
these bugs are still perceived by users.
The patches for these bugs are to aggregate I/O operations 
or completely eliminate unnecessary I/O operations.   
For example, Mozilla\#490742 is caused by bookmarking tabs 
by using separated database transactions. 
Even bookmarking 50 tabs can cause a timeout dialog window to pop up 
in a performance failure run.
To fix this bug, mozilla developers use one single aggregated transaction 
to bookmark all tabs.
As another example, Mozilla\#344059 is due to saving unchanged 
search engine preferences to SQLite, 
and it is fixed by only saving search 
engine preferences when some of them are changed.

For the other 3 bugs, their buggy loops will execute many iterations 
during performance failure runs,
and they are fixed by adding shortcuts to completely skip the buggy loops.
Take MySQL\#33948 as an example, 
originally MySQL developers keep both used and free table entries in the same linked list,
and the buggy loop is to iterate the linked list and look for a free entry.
During performance failure runs, many table entries are used and the buggy loop 
needs to take a lot of iterations to find a free entry. 
To fix this bug, MySQL developers simply separate used entries 
and free entries, and keep them in two different linked lists. 

\begin{figure}
\centering
\lstset{basicstyle=\ttfamily\fontsize{7}{8}\selectfont,
     morekeywords={+},keepspaces=true,numbers=left}
  \mbox{\lstinputlisting[mathescape,boxpos=t]{figure/mysql27287.c}}
\caption{A MySQL performance problem in polynomial complexity. 
\footnotesize{(During performance failure runs, 
   the total loop iterations scale polynomially in the size of \texttt{items}.)}}
\vspace{-0.05in}
\label{fig:mysql27287}
\vspace{-0.05in}
\end{figure}

{\underline{\textit{$O(N^k)$: polynomial complexity (k>1).}}}
As shown in Table~\ref{tab:study}, 
more than half of the studied complexity bugs are in polynomial complexity. 
Similar to complexity problems in linear complexity,
polynomial complexity for each problem is also caused by a buggy loop.
However, {\bf both} loop execution number and average loop iteration number
scale as input size $O(N)$.

To fix the majority (13/18) of performance problems in this category, 
developers directly modify the loops whose total loop iterations scale polynomially.  
The buggy loop for MySQL\#27287 is shown in Figure~\ref{fig:mysql27287}.
The loop searches parent \texttt{XML\_NODE} for function parameter \texttt{nitems}, 
which presents array index for another \texttt{XML\_NODE}.
All \texttt{XML\_NODE}s are maintained in array \texttt{items}. 
The way the loop to conduct the search is to iterate array \texttt{items} 
backward from the input and look for the first \texttt{XML\_NODE} 
whose level is one less than the input.
During performance failure runs, 
one \texttt{XML\_NODE} contains tens of thousands of children, 
and \texttt{xml\_parent\_tag} is invoked for each of its children. 
To fix this bug, MySQL developers add an extra field to each 
\texttt{XML\_NODE} to save its parent, 
and this field is initialized when a \texttt{XML\_NODE} is created. 
After that, the buggy loop is completely removed. 


To fix other performance problems (5/18), 
developers reduce data processed by the loops scaling polynomially, 
instead of optimizing the loops directly. 
For example, the loop scaling polynomially for GCC\#12322 is part 
of the functionality to do basic block reordering, 
and it will perform poorly for basic blocks with many successors. 
The bug-triggering input contains a basic block with thousands of successors. 
To fix this, developers simply skip basic blocks with many successors, 
which will lose some optimization opportunities, 
but can still keep generated codes working correctly. 


\begin{figure}
\centering
\lstset{basicstyle=\ttfamily\fontsize{7}{8}\selectfont,
     morekeywords={+},keepspaces=true,numbers=left}
  \mbox{\lstinputlisting[mathescape,boxpos=t]{figure/gcc27733.c}}
\caption{A GCC performance problem in exponential complexity. 
 \footnotesize{(During performance failure runs, how many times \texttt{mult\_alg} is invoked scales exponentially
  in the number of 1s in the binary form of input \texttt{t}.)}}
\vspace{-0.05in}
\label{fig:gcc27733}
\vspace{-0.05in}
\end{figure}


{\underline{\textit{$O(e^N)$: exponential complexity.}}}
4 studied complexity problems are in exponential complexity. 
These complexity problems are fixed by 
either leveraging memoization to reuse previous results 
or skipping computation with exponential complexity for large workload. 

3 of the performance problems are caused by recursive function calls. 
Take GCC\#27733 in Figure~\ref{fig:gcc27733} as an example, 
recursive function \texttt{mult\_alg} computes the best algorithm to multiply \texttt{t}.
In each invocation, \texttt{mult\_alg} will try a set of bitwise 
operations to change input 
\texttt{t} into a smaller number \texttt{t'} and recursively call itself. 
The number of times when \texttt{mult\_alg} is invoked scales exponentially 
in terms of 1s in the binary form of input \texttt{t}. 
To optimize this function, 
GCC developers use a hash table \texttt{alg\_hash} to record
which \texttt{t} has been processed before and what is its corresponding result.
However, there is a type declaration error inside the hash table entry,
and this error causes \texttt{t} larger than the maximum unsigned integer to never hit cache.
For large \texttt{t}, \texttt{mult\_alg} is still in exponential complexity. 
After fixing the type declaration error, 
memoization is enabled for large \texttt{t}. 

GCC\#32540 is in exponential complexity and is caused by an inefficient loop. 
The loop applies iterative algorithm to implement a compiler optimization. 
The bug-triggering input contains very complex control and data dependence,  
so that the buggy loop scales exponentially in terms of \texttt{if} branch in the input. 
To fix this bug, developers simply disable the optimization 
after detecting complex control and data dependence.  



\subsection{Diagnosis process of complexity problems}
\label{sec:process}

We also studied how complexity problems are reported by users and get diagnosed by developers. 

{\underline{\textit{How complexity problems are reported?}}
As shown in Table~\ref{tab:study},
for most complexity bugs (25/35), 
how to change input sizes in order to observe performance scaling problems 
is specified by users during reporting. 
As discussed in previous works~\cite{SongOOPSLA2014}, 
all collected user-perceived performance bug reports
contain bug-triggering inputs provided by users. 
For most complexity bugs, 
users will also describe how they change input sizes 
and how performance changes accordingly. 
For example, when reporting GCC\#32540, 
the user provides a C file and several measurement results to 
show that GCC compilation time scales exponentially 
with the number of \texttt{if} branch in the C file. 
As another example, when reporting Mozilla\#490742, 
the user mentions that the length of high, 
flat CPU usage is related to the number of search engine preferences. 

For the other 5 bugs, users provide one bug-triggering input 
or describe a set of bug-triggering inputs when they report the performance problems. 
For example, when reporting GCC\#27733, 
the user provides one bad input, which can take GCC 49 minutes to compile. 
When reporting Mozilla\#231300, the user mentions that clearing disk cache on Mac can trigger the bug, 
no matter what is the content of the cached files. 

{\underline{\textit{How complexity problems are diagnosed?}}
We study how long it takes developers to fix complexity problems. 
On average, it takes developers 162 days to diagnose a complexity problem. 
Following the same methodology as the previous work~\cite{SongOOPSLA2014},
we calculate diagnosis time starting from when a problem was reported 
and ending at when correct patches are submitted on bug tracking systems. 
On average, it takes developers 101 days to diagnose non-complexity bugs in the same set. 
Diagnosing complexity problems takes longer time. 


As discussed in the previous works~\cite{SongOOPSLA2014}, 
profiler is the only diagnosis tool mentioned in bug reports 
when diagnosing the collected user-perceived performance problems. 
Profiler is mentioned in 7 out of 30 complexity bug reports. 
The percentage of bug reports where profiler is mentioned 
is similar compared complexity bugs with non-complexity bugs. 


\subsection{Implications}


{\underline{\textit{Implication 1}}
For most studied complexity problems (25/30),
their patches are to directly optimize loops or recursive functions scaling poorly.
Accurately attributing complexity to loops or to functions can effectively guide developers where to investigate. 
Our study also shows that developers need to spend longer time to 
diagnose and fix complexity problems.
Traditional profiler is the only tool mentioned during diagnosis, 
and traditional profilers can only tell where computation is spent in each run, 
while can{\bf not} analyze or predict how computation scales across different runs.
To sum up, automatic algorithmic profilers are solely needed to effectively combat complexity problems.  

{\underline{\textit{Implication 2}}
In order to effectively profile algorithmic complexity,
a set of inputs with similar code coverage while different input sizes is needed. 
Our study shows that when reporting complexity problems,
users will describe how to change input sizes, 
while preserving similar functionalities (or code coverage). 
It is fairly easy for developers to get necessary inputs in order 
to conduct diagnosis or algorithmic profiling for user-reported complexity problems. 

 
{\underline{\textit{Implication 3}}
We also refer previous studying results to understand how complexity problems are introduced. 
20 out of 30 studied complexity bugs are introduced, 
because developers' workload assumption is wrong or workload has changed. 
This result motivate to monitor workload and conduct algorithmic profiling in production runs. 
Workload monitoring tools can warn developers when workload assumption is wrong.
Online algorithmic profiling tools can tell developers 
how each piece of codes scales under workload on the user side.  


{\underline{\textit{Implication 4}}}
Our study shows that around three-fourths of studied complexity problems (22/30)
are caused by repeated execution of buggy code constructs. 
These bugs are categorized as polynomial complexity and exponential complexity in Section~\ref{sec:tax}.
Previous works~\cite{SongOOPSLA2014,ldoctor} 
show that sampling code constructs 
executing many times can lower the runtime overhead 
while preserving the same diagnosis or detection capability. 
It is promising to apply sampling and design production-run algorithmic profiling techniques. 
