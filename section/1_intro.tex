\section{Introduction}
\label{sec:intro}

\subsection{Motivation}
\label{sec:motiv}

Performance problems\footnote{We will use performance problems and performance bugs exchangeably, 
following previous works in this area~\cite{SongOOPSLA2014,ldoctor}} 
are one type of software implementation mistakes
and can cause inefficient execution~\cite{PerfBug,perf.fse10,SongOOPSLA2014,ldoctor,Alabama}. 
Performance problems cannot be optimized away by state-of-the-art compiler optimizations.
Due to the complexity of modern software and rapidly changing workloads, 
performance problems widely exist in production-run software, 
annoying end users and wasting energy in the field~\cite{PerfBug,SongOOPSLA2014,ldoctor}. 
Many highly-publicized failures have already been caused by performance problems, 
such as making a website costing millions of dollars useless~\cite{ACA-health}.
Combating performance problems is urgent. 

Many performance problems are caused by algorithmic inefficiency, 
such as implementing a linear algorithm in a $O(N^2)$ way.
We refer these performance problems as complexity problems in our paper.
Our empirical study on a representative performance-bug 
benchmark set~\cite{PerfBug,SongOOPSLA2014} shows that 
nearly half of user-perceived performance problems are complexity problems. 
Complexity problems are usually ranked high in developers' priority list. 
For example, Mozilla developers will immediately try to fix complexity bugs degrading exponentially~\cite{mozilla35294}. 
Addressing complexity problems is an important aspect to fight performance bugs. 


Algorithmic profiling~\cite{Aprof1,Aprof2,AlgoProf} collects profiles from multiple 
runs of the same program and attributes complexity to different code constructs, such as a loop or a function,
in the form of a \textit{cost} function of \textit{input} size. 
Algorithmic profiling can be used detect previously unknown complexity problems and 
diagnose performance failures caused by complexity problems. 
Conducting algorithmic profiling is challenging. 
Effective techniques need to satisfy the following three requirements.

\begin{itemize}

\item First, \textit{coverage}. Complexity problems are in different types of complexity, 
and they are also caused by a large variety of root causes. 
Given a code construct, it may take inputs and 
consume computation resources in various types.
A good algorithmic profiling technique must cover a 
large portion of various complexity types, root causes, inputs and costs.


\item Second, \textit{accuracy}. 
Given an analyzed code construct,
algorithmic profiling needs accurately identify 
how cost scales as input size changes.
It is desired to conduct algorithmic profiling under the context of the whole program's execution. 
Otherwise, missing how the code construct cooperates with other parts will lead to inaccurate results. 

\item Third, \textit{performance}. 
Lower runtime overhead can allow developers to run more tests 
under a given time budget during in-house testing.
Production-run algorithmic profiling can help developers 
understand how their programs scale
under real-world workloads.
To be deployed in production runs, 
techniques must not incur any observable slowdown.


\end{itemize}

Existing techniques do not satisfy the above three requirements and 
cannot conduct algorithmic profiling effectively. 
Traditional profilers are the most widely used tools to 
diagnose performance failures~\cite{gprof,oprofile}. 
Traditional profilers can only measure how much time spent in each code construct during one single run, 
while failing to connect information from multiple runs 
and failing to provide any indication about how execution time scales.
Therefore, traditional profilers fail in both coverage and accuracy.  
To understand asymptotic complexity for a code construct,
experimental algorithmics~\cite{expalg1,expalg2,expalg3} requires developers to 
extract the code construct and analyze it out of the original program. 
Although useful,
experimental algorithmics fail to consider how the code 
construct interact with the whole program and does not provide desired accuracy. 
Recent techniques on algorithmic profiling~\cite{Aprof1,Aprof2,AlgoProf} will incur more than $30$X runtime overhead.
They cannot provide desired performance and are far from being deployed in production runs. 


\begin{figure}
\centering
\lstset{basicstyle=\ttfamily\fontsize{7}{8}\selectfont,
     morekeywords={+},keepspaces=true,numbers=left}
  \mbox{\lstinputlisting[mathescape,boxpos=t]{figure/mysql27287.c}}
\caption{A MySQL performance problem in polynomial complexity. 
\footnotesize{(This figure shows the buggy code fragment for MySQL\#27287. 
   During performance failure runs, 
   execution time scales polynomially in the size of \texttt{items}.)}}
\vspace{-0.05in}
\label{fig:mysql27287}
\vspace{-0.05in}
\end{figure}


One MySQL complexity problem is shown in Figure~\ref{fig:mysql27287}.
The loop searches parent \texttt{XML\_NODE} for function parameter \texttt{nitems}, 
which presents an array index for another \texttt{XML\_NODE}.
All \texttt{XML\_NODE}s are maintained in array \texttt{items}. 
The loop searches by iterating array \texttt{items} 
backward from the input and looking for the first \texttt{XML\_NODE} 
whose level is one less than the input.
This piece of codes looks innocent. 
However, there is an outer loop not shown in the figure.
The outer loop will keep calling \texttt{xml\_parent\_tag} by using 
the next sibling of the previous \texttt{XML\_NODE}, 
which is in $O(N^2)$ complexity in terms of the number of children of a parent \texttt{XML\_NODE}. 
Developers may think using an implementation in $O(N^2)$ complexity is fine, 
since a \texttt{XML\_NODE} usually does not have too many children.
However, during performance failure runs, 
one \texttt{XML\_NODE} contains tens of thousands of children, 
and this leads significant showdown perceived by end users. 

To fix this bug, developers add an extra field to each \texttt{XML\_NODE} to save its parent, 
and this field is initialized when a \texttt{XML\_NODE} is created. 
After applying this patch, codes shown in Figure~\ref{fig:mysql27287} are completely removed.
This takes developers around 5 months to figure 
out\footnote{We count the time from when developers confirmed this is performance bug 
to when the patch was submitted}. 
In-house algorithmic profiling can warn developers the $O(N^2)$ complexity, 
potentially obscured by the fact that inner and outer loops are not in the same function. 
Production-run algorithmic profiling can provide information about the real-world workload 
processed by this piece of code and its corresponding complexity,
which are invaluable to diagnose performance failures caused by this complexity problem. 


\subsection{Our Contributions}
\label{sec:con}

In this paper, 
we present a toolchain \Tool to effectively conduct algorithmic profiling with 
good coverage, accuracy, and performance. 
\Tool has in-house and production-run two settings. 
Under the in-house setting, 
\Tool is launched by using existing tests, tests from end users, 
or tests generated by test input generation techniques~\cite{KLEE,s2e,dart,EventBreak}. 
\Tool will automatically merge profiles from multiple runs and 
identify code constructs with super-linear complexity. 
Under the production-run setting, 
instrumented programs will be distributed to end users 
and complexity information will be collected in negligible runtime overhead. 
Developers can leverage this information to better understand real-world workloads
and diagnose performance failures caused by complexity problems. 


We build \Tool through the following three steps.

We first conduct an empirical study to better understand 
real-world complexity problems.
All studied complexity bugs come from a representative 
performance-bug benchmark suite~\cite{PerfBug,SongOOPSLA2014}.
To the best of our knowledge, our work is the first study focusing on complexity problems.
We categorize complexity problems into different complexity categories.
We study root causes, 
how user-perceived performance impact is generated, 
and fix strategies for bugs in each category. 
We also investigate the reporting and diagnosis process for complexity problems.
Our findings and implications can motivate future research on complexity problems. 
They have already guided our design point selection when building the in-house version of \Tool, 
and inspired us to apply sampling to the production-run version of \Tool. 

We then systematically explore different design points during algorithmic profiling 
in order to build the in-house version of \Tool. 
The design points include two types of input metrics, 
three types of cost metrics, 
one method to fit cost curve, 
and one method to compare different cost curves. 
Our experience shows that naive implementations 
to collect runtime information
will lead to extremely large overhead. 
We design several optimizations to accelerate information collection 
and avoid collecting redundant information.  
We evaluate different design points and our optimizations 
by using $46$ complexity problems from three sources. 
Our evaluation results show that after choosing the correct design points, 
\Tool can accurately attribute complexity to code constructs 
leading to the perceived performance failures. 

To build the production-run version of \Tool,
we apply software-based sampling to algorithmic profiling.
Given a code construct with multiple dynamic instances in a program run,
we sample some of the dynamic instances and leverage the mark-and-recapture 
method~\citep{mark-recapture} to infer information for all instances. 
Our evaluation results show that 
sampling allows us to significantly lower the runtime overhead.
The results also show that sampling does not hurt the profiling capability 
and does not require to profile more program runs, 
not increasing profiling latency, due to the fact that
the majority of complexity problems are caused by 
repeated executions of a loop or a recursive function as shown by our empirical study.



Specifically, we make the following contributions:

\begin{itemize}

\item First, we conduct the first empirical study on real-world complexity problems. 
We have several important findings and implications, including
1) around three fourths of the studied complexity problems are 
caused by repeated executions of a loop or a recursive function;
2) for most complexity problems, 
users describe how to change input size to observe the scaling problem during reporting;
and 3) complexity problems usually take longer time to diagnose and fix, 
and more effective tool supports are needed.  

\item Second, we design and implement the in-house version of \Tool through 
thoroughly investigating different design points during algorithmic profiling. 
Experimental results show that \Tool can effectively analyze performance failures 
caused by different types of complexity problems and attribute accurate complexity information.  

\item Third, we design the production-run version of \Tool by applying 
software-based sampling and a statistical estimation method. 
Experimental results show that we can lower the runtime overhead to less than 
$5\%$ for $36$ out of $46$ benchmarks, 
without losing profiling capability and increasing profiling latency.  


\end{itemize}

